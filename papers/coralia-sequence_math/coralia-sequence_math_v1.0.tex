\documentclass[11pt]{article}
\usepackage{amsmath,amsthm,amssymb,booktabs,hyperref}
\usepackage[margin=1in]{geometry}

\newtheorem{theorem}{Theorem}
\newtheorem{definition}{Definition}

\title{The Coralia Sequence: A Unique Finite Integer Set\\Under Fibonacci-Lucas Constraints}
\author{Emma Cecile\\ORCID: 0009-0008-4120-9309\\Coralia LLC}
\date{January 2026}

\begin{document}
\maketitle

\begin{abstract}
We introduce the Coralia Sequence $\mathcal{C} = \{0,1,2,3,5,7,9,12,15,23,30,35\}$, a 12-element integer set generated by zone-based construction. We prove $\mathcal{C}$ is the unique completion of a 10-element seed under axioms C1--C9.
\end{abstract}

\section{Introduction}
The Coralia Sequence bridges Fibonacci and Lucas structures through constrained generation.

\begin{definition}[Zone Generator]
Four zones produce $\mathcal{C}$: gaps $(1,1,1)$, $(2,2,2)$, $(3,3)$, $(8,7,5)$.
\end{definition}

\begin{theorem}[Uniqueness]
Axioms C1--C9 fix seed $\{0,1,2,3,5,7,9,12,15,35\}$. Only $(23,30)$ complete it.
\end{theorem}

\begin{theorem}[Terminal Uniqueness]
$(8,7,5)$ is the unique descending triple from Fib $\cup$ Luc summing to 20 with all elements $\geq 5$.
\end{theorem}

\begin{thebibliography}{9}
\bibitem{koshy} T.~Koshy, \emph{Fibonacci and Lucas Numbers with Applications}, Wiley, 2001.
\end{thebibliography}

\end{document}
