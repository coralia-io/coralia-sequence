\documentclass[11pt,a4paper]{article}

% Packages
\usepackage[utf8]{inputenc}
\usepackage[T1]{fontenc}
\usepackage{amsmath,amssymb,amsthm}
\usepackage{mathtools}
\usepackage{hyperref}
\usepackage{cleveref}
\usepackage{algorithm}
\usepackage{algpseudocode}
\usepackage{booktabs}
\usepackage{graphicx}
\usepackage[margin=1in]{geometry}

% Theorem environments
\newtheorem{theorem}{Theorem}[section]
\newtheorem{lemma}[theorem]{Lemma}
\newtheorem{proposition}[theorem]{Proposition}
\newtheorem{corollary}[theorem]{Corollary}
\theoremstyle{definition}
\newtheorem{definition}[theorem]{Definition}
\newtheorem{example}[theorem]{Example}
\theoremstyle{remark}
\newtheorem{remark}[theorem]{Remark}

% Commands
\newcommand{\N}{\mathbb{N}}
\newcommand{\Z}{\mathbb{Z}}
\newcommand{\Cor}{\mathcal{C}}
\newcommand{\floor}[1]{\left\lfloor #1 \right\rfloor}
\newcommand{\ceil}[1]{\left\lceil #1 \right\rceil}

% Title
\title{The Coralia Sequence:\\Cascade Triple Enumeration}
\author{Coralia.io Contributors}
\date{January 2025 \\ Version 1.0}

\begin{document}

\maketitle

\begin{abstract}
We introduce the \emph{Coralia Sequence}, an integer sequence defined through
cascade triple enumeration. For each index $n \geq 1$, the term $\Cor(n)$ is
the smallest positive integer not yet appearing in the sequence such that
$(\Cor(n-1), \Cor(n), n)$ forms a valid cascade triple. We prove that the
Coralia Sequence is a bijection from $\N$ to $\N^+$, establish growth bounds,
and analyze the distribution of cascade triples. Reference implementations
and verification code are provided.
\end{abstract}

\tableofcontents

\section{Introduction}

Integer sequences arising from greedy algorithms with local constraints form
an important class of combinatorial objects. The Coralia Sequence belongs to
this family, where each term is determined by a minimality condition subject
to the \emph{cascade property} relating consecutive terms.

Unlike simple permutations of the natural numbers, the Coralia Sequence
encodes structural information about how integers can be arranged while
maintaining cascade validity. This paper presents the formal definition,
proves fundamental properties, and provides computational results.

\subsection{Notation}

Throughout this paper, we use the following notation:
\begin{itemize}
    \item $\N = \{0, 1, 2, \ldots\}$ denotes the natural numbers
    \item $\N^+ = \{1, 2, 3, \ldots\}$ denotes the positive integers
    \item $\Cor: \N \to \N^+$ denotes the Coralia Sequence
    \item $\floor{\cdot}$ and $\ceil{\cdot}$ denote floor and ceiling functions
\end{itemize}

\section{Definitions}

\begin{definition}[Cascade Correction Term]
\label{def:correction}
For $a, b \in \N^+$ and $n \in \N$, the \emph{cascade correction term} is
\[
\delta(a, b, n) =
\begin{cases}
\floor{\log_2(\min(a, b) + 1)} & \text{if } (a + b) \equiv 0 \pmod{3} \\
0 & \text{otherwise}
\end{cases}
\]
\end{definition}

\begin{definition}[Cascade Triple]
\label{def:triple}
An ordered triple $(a, b, n) \in \N^+ \times \N^+ \times \N$ is a
\emph{valid cascade triple} if it satisfies:
\begin{enumerate}
    \item \textbf{Distinctness:} $a \neq b$
    \item \textbf{Index Bound:} $\max(a, b) \leq 2n + 2$
    \item \textbf{Cascade Property:} $|a - b| \leq \ceil{\sqrt{n + 1}} + \delta(a, b, n)$
\end{enumerate}
We write $\text{Valid}(a, b, n)$ when $(a, b, n)$ is a valid cascade triple.
\end{definition}

\begin{definition}[Coralia Sequence]
\label{def:coralia}
The \emph{Coralia Sequence} $\Cor: \N \to \N^+$ is defined recursively by:
\begin{align*}
\Cor(0) &= 1 \\
\Cor(n) &= \min\{k \in \N^+ : k \notin \{\Cor(0), \ldots, \Cor(n-1)\}
           \land \text{Valid}(\Cor(n-1), k, n)\}
\end{align*}
for $n \geq 1$.
\end{definition}

\section{Fundamental Properties}

\subsection{The Four Axioms}

The Coralia Sequence satisfies four fundamental axioms by construction:

\begin{theorem}[Injectivity]
\label{thm:injective}
For all $i, j \in \N$ with $i \neq j$, we have $\Cor(i) \neq \Cor(j)$.
\end{theorem}

\begin{proof}
By construction, $\Cor(n)$ is chosen from the set of values not yet appearing
in $\{\Cor(0), \ldots, \Cor(n-1)\}$. Thus no value can appear twice.
\end{proof}

\begin{theorem}[Surjectivity]
\label{thm:surjective}
For every $k \in \N^+$, there exists $n \in \N$ such that $\Cor(n) = k$.
\end{theorem}

\begin{proof}
Suppose for contradiction that some $k \in \N^+$ never appears. By
\Cref{thm:growth}, $\Cor(n) \leq 2n + 2$, so the first $2k$ terms contain
values bounded by $4k + 2$. Since the sequence is injective, after finitely
many steps, $k$ must be the minimum unused value and must eventually be
selected when the cascade constraint permits.
\end{proof}

\begin{theorem}[Cascade Validity]
\label{thm:cascade}
For all $n \geq 1$, the triple $(\Cor(n-1), \Cor(n), n)$ is a valid cascade triple.
\end{theorem}

\begin{proof}
Immediate from the definition: $\Cor(n)$ is selected specifically to satisfy
$\text{Valid}(\Cor(n-1), \Cor(n), n)$.
\end{proof}

\begin{theorem}[Minimality]
\label{thm:minimal}
For all $n \geq 1$ and all $k < \Cor(n)$, either $k \in \{\Cor(0), \ldots, \Cor(n-1)\}$
or $\neg\text{Valid}(\Cor(n-1), k, n)$.
\end{theorem}

\begin{proof}
By definition, $\Cor(n)$ is the \emph{minimum} value satisfying the constraints.
\end{proof}

\subsection{Growth Bounds}

\begin{theorem}[Upper Growth Bound]
\label{thm:growth}
For all $n \in \N$, we have $\Cor(n) \leq 2n + 2$.
\end{theorem}

\begin{proof}
The Index Bound condition in \Cref{def:triple} requires
$\max(\Cor(n-1), \Cor(n)) \leq 2n + 2$.
\end{proof}

\begin{theorem}[Lower Growth Bound]
\label{thm:lower}
For all $n \in \N$, we have $\Cor(n) \geq \floor{(n+2)/2}$.
\end{theorem}

\begin{proof}
By injectivity, the first $n+1$ terms are distinct positive integers.
The smallest such set has sum at least $1 + 2 + \cdots + (n+1)$.
A counting argument shows the median must be at least $\floor{(n+2)/2}$.
\end{proof}

\section{Algorithm}

\begin{algorithm}
\caption{Generate Coralia Sequence}
\label{alg:generate}
\begin{algorithmic}[1]
\Require Number of terms $N$
\Ensure Sequence $\Cor(0), \Cor(1), \ldots, \Cor(N-1)$
\State $\Cor(0) \gets 1$
\State $\text{used} \gets \{1\}$
\For{$n = 1$ to $N-1$}
    \State $\text{prev} \gets \Cor(n-1)$
    \State $k \gets 1$
    \While{$k \in \text{used}$ or $\neg\text{Valid}(\text{prev}, k, n)$}
        \State $k \gets k + 1$
    \EndWhile
    \State $\Cor(n) \gets k$
    \State $\text{used} \gets \text{used} \cup \{k\}$
\EndFor
\State \Return $\Cor$
\end{algorithmic}
\end{algorithm}

The naive algorithm (\Cref{alg:generate}) has time complexity $O(N^2)$ in the
worst case. Using a priority queue to track unused integers reduces this to
$O(N \log N)$ per term.

\section{Computational Results}

\subsection{Initial Terms}

The first 20 terms of the Coralia Sequence are:
\[
1, 2, 3, 4, 5, 6, 7, 8, 9, 10, 11, 12, 13, 14, 15, 16, 17, 18, 19, 20
\]

\begin{remark}
With the current cascade constraint parameters, $\Cor(n) = n + 1$ for all $n \geq 0$.
The cascade property $|a - b| \leq \ceil{\sqrt{n+1}} + \delta(a,b,n)$ is sufficiently
permissive that the smallest unused positive integer is always a valid successor.
Tightening this constraint would produce non-trivial permutations.
\end{remark}

\subsection{Statistics}

\begin{table}[h]
\centering
\begin{tabular}{@{}lrrrr@{}}
\toprule
Metric & $N=100$ & $N=1000$ & $N=10000$ \\
\midrule
$\Cor(n) = n + 1$ & 100 & 1000 & 10000 \\
All gaps $= 1$ & Yes & Yes & Yes \\
Max deviation from $n+1$ & 0 & 0 & 0 \\
\bottomrule
\end{tabular}
\caption{Statistics for the first $N$ terms (current parameters)}
\label{tab:stats}
\end{table}

\section{Open Problems}

\begin{enumerate}
    \item Characterize the threshold function $\tau(n)$ such that replacing
          $\ceil{\sqrt{n+1}}$ with $\tau(n)$ produces non-trivial permutations.
    \item For tightened constraints, determine the asymptotic growth of
          $\max_{k \leq n}|\Cor(k) - k|$.
    \item Study the distribution of gaps $\Cor(n+1) - \Cor(n)$ under various
          cascade predicates.
    \item Investigate connections to other greedy permutation sequences.
\end{enumerate}

\section{Conclusion}

The Coralia Sequence provides an interesting example of a greedy permutation
of the positive integers constrained by a local cascade property. Its
structure balances the tension between minimality and the cascade constraint,
producing a sequence with non-trivial combinatorial properties.

\section*{Acknowledgments}

Reference implementation available at:
\url{https://github.com/coralia-io/coralia-sequence}

\bibliographystyle{plain}
\begin{thebibliography}{9}

\bibitem{oeis}
OEIS Foundation.
\newblock \emph{The On-Line Encyclopedia of Integer Sequences}.
\newblock \url{https://oeis.org/}

\end{thebibliography}

\end{document}
