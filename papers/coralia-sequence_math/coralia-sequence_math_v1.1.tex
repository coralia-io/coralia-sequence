\documentclass[11pt,a4paper]{article}

\usepackage{amsmath,amsthm,amssymb}
\usepackage{mathtools}
\usepackage{enumitem}
\usepackage{booktabs}
\usepackage{hyperref}
\usepackage[margin=1in]{geometry}

\newtheorem{theorem}{Theorem}[section]
\newtheorem{proposition}[theorem]{Proposition}
\newtheorem{lemma}[theorem]{Lemma}
\newtheorem{corollary}[theorem]{Corollary}
\theoremstyle{definition}
\newtheorem{definition}[theorem]{Definition}
\theoremstyle{remark}
\newtheorem{remark}[theorem]{Remark}

\newcommand{\C}{\mathcal{C}}
\newcommand{\gaps}{\mathrm{gaps}}
\newcommand{\Fib}{\mathrm{Fib}}
\newcommand{\Luc}{\mathrm{Luc}}
\newcommand{\FL}{\mathrm{FL}}

\title{The Coralia Sequence: A Unique Finite Integer Set\\ Under Fibonacci--Lucas Terminal Constraints}

\author{Emma Cecile\\
\small ORCID: 0009-0008-4120-9309\\
\small Coralia LLC}

\date{January 2026}

\begin{document}

\maketitle

\begin{abstract}
We introduce the Coralia Sequence $\C = \{0, 1, 2, 3, 5, 7, 9, 12, 15, 23, 30, 35\}$, a 12-element integer set characterized by nine axioms (C1--C9). The sequence is generated by four arithmetic zones with ascending gaps followed by a descending Fibonacci--Lucas cascade. We prove that $\C$ is the unique 12-element completion of a 10-element seed determined by the axioms. The terminal gap triple $(8, 7, 5) = (F_6, L_4, F_5)$ is shown to be the unique strictly descending triple from $\Fib \cup \Luc$ summing to 20 with all elements at least 5.
\end{abstract}

%==============================================================================
\section{Introduction}
%==============================================================================

Let $\varphi = (1 + \sqrt{5})/2$ denote the golden ratio. This paper introduces a finite integer set that incorporates Fibonacci--Lucas structure in its terminal gap constraints.

\begin{definition}[Fibonacci and Lucas sequences]
The Fibonacci sequence $(F_n)_{n \geq 0}$ and Lucas sequence $(L_n)_{n \geq 0}$ are defined by:
\begin{align*}
F_0 &= 0, \quad F_1 = 1, \quad F_n = F_{n-1} + F_{n-2} \text{ for } n \geq 2, \\
L_0 &= 2, \quad L_1 = 1, \quad L_n = L_{n-1} + L_{n-2} \text{ for } n \geq 2.
\end{align*}
We write $\Fib = \{F_n : n \geq 0\}$ and $\Luc = \{L_n : n \geq 0\}$ for the sets of Fibonacci and Lucas numbers, respectively. Note that $\Fib \cap \Luc = \{1, 3\}$ and $4 \in \Luc$ (as $L_3 = 4$).
\end{definition}

\begin{definition}[Ordered set and gap sequence]
\label{def:gaps}
Let $S \subset \mathbb{Z}$ be a finite set of integers with $|S| = n$. We write $S = \{s_0 < s_1 < \cdots < s_{n-1}\}$ for its strictly increasing enumeration. The \emph{gap sequence} of $S$ is
\[
\gaps(S) = (g_0, g_1, \ldots, g_{n-2}) \quad \text{where} \quad g_i = s_{i+1} - s_i.
\]
We refer to $g_0, \ldots, g_7$ as the \emph{first eight gaps} and $g_{n-4}, g_{n-3}, g_{n-2}$ as the \emph{last three gaps} (when $n \geq 4$).
\end{definition}

\begin{lemma}[Telescoping sum]
\label{lem:telescope}
For any finite set $S = \{s_0 < \cdots < s_{n-1}\}$,
\[
\sum_{i=0}^{n-2} g_i = s_{n-1} - s_0 = \max(S) - \min(S).
\]
\end{lemma}

\begin{proof}
Direct telescoping: $\sum_{i=0}^{n-2} (s_{i+1} - s_i) = s_{n-1} - s_0$.
\end{proof}

%==============================================================================
\section{The Axiom System}
%==============================================================================

Let $\C \subseteq \{0, 1, \ldots, 35\}$ with $|\C| = 12$. Write $\C = \{c_0 < c_1 < \cdots < c_{11}\}$ and $\gaps(\C) = (g_0, \ldots, g_{10})$.

\medskip
\noindent\textbf{Axioms C1--C9:}
\begin{itemize}[leftmargin=2cm]
    \item[\textbf{C1.}] (Origin) $0 \in \C$.
    \item[\textbf{C2.}] (Cardinality containment) $|\C| \in \C$.
    \item[\textbf{C3.}] (Cardinality) $|\C| = 12$.
    \item[\textbf{C4.}] (Ceiling) $\max(\C) = 35$.
    \item[\textbf{C5.}] (Gap-sum closure; derivable from C1, C4) $\sum_{i=0}^{10} g_i = 35$.
    \item[\textbf{C6.}] (Foundation structure) The first eight gaps satisfy:
    \begin{itemize}
        \item[(a)] $g_0, g_1, \ldots, g_7 \in \{1, 2, 3\}$,
        \item[(b)] $g_0 \leq g_1 \leq \cdots \leq g_7$,
        \item[(c)] (Seed specification) $\{1, 2, 3, 5, 7, 9, 15\} \subseteq \C$.
    \end{itemize}
    \item[\textbf{C7.}] (Terminal descent) $g_8 > g_9 > g_{10}$.
    \item[\textbf{C8.}] (Fibonacci--Lucas terminal) $g_8, g_9, g_{10} \in \Fib \cup \Luc$.
    \item[\textbf{C9.}] (Terminal minimum) $g_8, g_9, g_{10} \geq 5$.
\end{itemize}

\begin{remark}
Axiom C6 has three subclauses (a)--(c); we count it as one axiom, giving nine axioms total. Axioms C2 and C3 together imply $12 \in \C$; we state them separately to distinguish the structural principle ($|\C| \in \C$) from the instantiated cardinality ($|\C| = 12$). By Lemma~\ref{lem:telescope}, C5 is equivalent to $\max(\C) - \min(\C) = 35$; given C1 and C4, this is automatic. We retain C5 to emphasize the gap-sum structure.
\end{remark}

%==============================================================================
\section{The Zone Generator}
%==============================================================================

\begin{definition}[Zone generator]
Define four zones $Z_1, Z_2, Z_3, Z_4$ as follows:
\begin{center}
\begin{tabular}{cccl}
\toprule
Zone & Start & Gap vector & Elements produced \\
\midrule
$Z_1$ & 0 & $(1, 1, 1)$ & $0, 1, 2, 3$ \\
$Z_2$ & 3 & $(2, 2, 2)$ & $5, 7, 9$ \\
$Z_3$ & 9 & $(3, 3)$ & $12, 15$ \\
$Z_4$ & 15 & $(8, 7, 5)$ & $23, 30, 35$ \\
\bottomrule
\end{tabular}
\end{center}
The Coralia Sequence is $\C = Z_1 \cup Z_2 \cup Z_3 \cup Z_4 = \{0, 1, 2, 3, 5, 7, 9, 12, 15, 23, 30, 35\}$.
\end{definition}

\begin{remark}
The zone generator produces $\C$ by construction. The axioms characterize $\C$. We prove below that the axioms uniquely determine $\C$.
\end{remark}

%==============================================================================
\section{Uniqueness Theorem}
%==============================================================================

\begin{lemma}[Forced seed]
\label{lem:seed}
Axioms C1, C2, C3, C4, and C6(c) together force the inclusion of a 10-element seed:
\[
S = \{0, 1, 2, 3, 5, 7, 9, 12, 15, 35\}.
\]
\end{lemma}

\begin{proof}
\begin{itemize}
    \item C1 forces $0 \in \C$.
    \item C4 forces $35 \in \C$ (as maximum).
    \item C6(c) forces $\{1, 2, 3, 5, 7, 9, 15\} \subseteq \C$.
    \item C2 and C3 force $12 \in \C$ (since $|\C| = 12$ must be a member).
\end{itemize}
These nine elements are distinct. Together with $35$, we have $|S| = 10$.
\end{proof}

\begin{lemma}[Completion count]
\label{lem:completion}
Any set $\C$ satisfying C1--C4 and C6(c) must be a 12-element subset of $\{0, 1, \ldots, 35\}$ containing the seed $S$. Exactly two elements must be added to $S$ from the pool $P = \{0, \ldots, 35\} \setminus S$, giving $\binom{26}{2} = 325$ candidate completions.
\end{lemma}

\begin{proof}
C3 fixes $|\C| = 12$. C4 fixes $\max(\C) = 35$. By C1, $\min(\C) = 0$. Thus $\C \subseteq \{0, \ldots, 35\}$. Since $|S| = 10$ and $|\C| = 12$, exactly two elements from $P$ must be added. The pool has $36 - 10 = 26$ elements.
\end{proof}

\begin{theorem}[Uniqueness]
\label{thm:uniqueness}
The Coralia Sequence $\C = \{0, 1, 2, 3, 5, 7, 9, 12, 15, 23, 30, 35\}$ is the unique 12-element subset of $\{0, \ldots, 35\}$ satisfying axioms C1--C9.
\end{theorem}

\begin{proof}
By Lemmas~\ref{lem:seed} and~\ref{lem:completion}, any solution must be a completion of the seed $S$ by exactly two elements from the 26-element pool $P$.

We verify uniqueness by exhaustive enumeration. For each of the $\binom{26}{2} = 325$ candidate pairs $\{a, b\} \subset P$, we form $\C' = S \cup \{a, b\}$, compute its ordered enumeration and gap sequence, and check axioms C5--C9:
\begin{itemize}
    \item C5: $\sum g_i = 35$,
    \item C6(a): $g_0, \ldots, g_7 \in \{1, 2, 3\}$,
    \item C6(b): $g_0 \leq g_1 \leq \cdots \leq g_7$,
    \item C7: $g_8 > g_9 > g_{10}$,
    \item C8: $g_8, g_9, g_{10} \in \Fib \cup \Luc$,
    \item C9: $g_8, g_9, g_{10} \geq 5$.
\end{itemize}

Exactly one completion passes all checks: $\{23, 30\}$, yielding $\C = \{0, 1, 2, 3, 5, 7, 9, 12, 15, 23, 30, 35\}$. Since the candidate space is finite (325 cases), exhaustive verification constitutes a complete proof.

\medskip
\noindent\textit{Verification.} The enumeration is implemented in Python and available at:

\centerline{\url{https://github.com/coralia-io/coralia-sequence}}

\noindent Running \texttt{verify\_uniqueness.py} confirms: \texttt{Solutions: 1}.
\end{proof}

%==============================================================================
\section{Terminal Triple Uniqueness}
%==============================================================================

\begin{theorem}[Terminal triple uniqueness]
\label{thm:terminal}
Let $\FL = (\Fib \cup \Luc) \cap \{1, 2, \ldots, 20\}$. The triple $(8, 7, 5)$ is the unique strictly descending triple $(a, b, c)$ with $a, b, c \in \FL$, $a + b + c = 20$, and $c \geq 5$.
\end{theorem}

\begin{proof}
We have $\FL = \{1, 2, 3, 4, 5, 7, 8, 11, 13, 18\}$.

Since $a > b > c \geq 1$ and $a + b + c = 20$, we have $a \geq 8$. The candidates $a \in \FL \cap \{8, 11, 13, 18\}$ yield:

\begin{center}
\begin{tabular}{cl}
\toprule
$a$ & Descending triples $(a, b, c)$ with $b + c = 20 - a$ \\
\midrule
18 & None (would need $b + c = 2$ with $b > c \geq 1$) \\
13 & $(13, 5, 2)$, $(13, 4, 3)$ \\
11 & $(11, 8, 1)$, $(11, 7, 2)$, $(11, 5, 4)$ \\
8 & $(8, 7, 5)$ \\
\bottomrule
\end{tabular}
\end{center}

All six triples are valid under the first three constraints. Since $a > b > c$, we have $\min(a, b, c) = c$. Applying $c \geq 5$, only $(8, 7, 5)$ survives.

The Fibonacci--Lucas identities are: $8 = F_6$, $7 = L_4$, $5 = F_5$.
\end{proof}

%==============================================================================
\section{Structural Properties}
%==============================================================================

\begin{proposition}[Closure properties]
The Coralia Sequence satisfies:
\begin{enumerate}[label=(\roman*)}
    \item \textbf{Cardinality containment:} $|\C| = 12 \in \C$.
    \item \textbf{Gap-sum closure:} $\sum \gaps(\C) = \max(\C) = 35$.
    \item \textbf{Cumulative generation:} The cumulative sums of $\gaps(\C)$ are $\C \setminus \{0\}$.
\end{enumerate}
\end{proposition}

\begin{proof}
Direct verification: $\gaps(\C) = (1, 1, 1, 2, 2, 2, 3, 3, 8, 7, 5)$.
\begin{enumerate}[label=(\roman*)]
    \item $12 \in \C$ by inspection.
    \item $1+1+1+2+2+2+3+3+8+7+5 = 35 = \max(\C)$.
    \item Cumulative sums: $1, 2, 3, 5, 7, 9, 12, 15, 23, 30, 35 = \C \setminus \{0\}$. \qedhere
\end{enumerate}
\end{proof}

%==============================================================================
\section{Open Questions}
%==============================================================================

\begin{enumerate}
    \item Does a natural extension of $\C$ exist beyond 35 preserving the axiom structure?
    \item Is there a closed-form characterization of the two free elements (23, 30)?
    \item Do analogous sets exist for other terminal triples from $\Fib \cup \Luc$?
\end{enumerate}

%==============================================================================
% References
%==============================================================================

\begin{thebibliography}{9}

\bibitem{koshy}
T.~Koshy, \emph{Fibonacci and Lucas Numbers with Applications}, Wiley, 2001.

\bibitem{vorobiev}
N.~N.~Vorobiev, \emph{Fibonacci Numbers}, Birkh\"auser, 2002.

\bibitem{oeis}
OEIS Foundation, \emph{The On-Line Encyclopedia of Integer Sequences}, \url{https://oeis.org}.

\end{thebibliography}

\end{document}
